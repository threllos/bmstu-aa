{\centering \chapter*{ВВЕДЕНИЕ}}
\addcontentsline{toc}{chapter}{ВВЕДЕНИЕ}

Задача коммивояжера находится в центре внимания с 1960 года. 
Суть ее состоит в том, чтобы найти кратчайший круговой маршрут, включающий посещение определенного числа $n$ вершин, причем начальная и конечная вершины являются одинаковыми, и каждая последующая вершина входит в этот круговой маршрут один раз \cite{Brezina2020}. 

Эта задача имеет большое количество практических приложений, особенно в сфере логистики (например, утилизация бытовых отходов, доставка товаров со склада, раздача хлебобулочных изделий из пекарен в отдельные магазины, планирование маршрута школьного автобуса, планирование услуг в компаниях, службы доставки, сверление отверстий под печатные платы, компьютерные системы, управление промышленными роботами, оптимизация схем, проектирование сетей и многое другое) \cite{Brezina2020}.

В последние два десятилетия при оптимизации сложных систем исследователи все чаще применяют природные механизмы поиска наилучших решений. 
Эти механизмы обеспечивают эффективную адаптацию флоры и фауны к окружающей среде на протяжении миллионов лет. 
Муравьиные алгоритмы серьезно исследуются европейскими учеными с середины 90-х годов. 
На сегодня уже получены хорошие результаты муравьиной оптимизации таких сложных комбинаторных задач, как: задачи коммивояжера, задачи оптимизации маршрутов грузовиков, задачи раскраски графа, квадратичной задачи о назначениях, оптимизации сетевых графиков, задачи календарного планирования и других \cite{Shtovba2003}. 