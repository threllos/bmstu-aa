\chapter*{Введение}
\addcontentsline{toc}{chapter}{Введение}

Нахождение редакционного расстояния -- одна из задач компьютерной лингвистики, которая находит применение в огромном количестве областей, например:
\begin{itemize}
	\item исправление ошибок в тексте в поисковых запросах;
    \item сравнение текстовых файлов утилитой diff \cite{diff};
    \item сравнение генов, хромосом и белков в биоинформатике.
\end{itemize}

Впервые задачу поставил советский ученый В. И. Левенштейн при изучении последовательностей 0-1 \cite{levenshtein}.
Впоследствии более общую задачу для произвольного алфавита связали с его именем.
Позже Фредерик Дамерау заявил, что при исследовании орфографических ошибок в информационно-поисковых системах более 80\% человеческих ошибок при наборе текстов составляют перестановки соседних символов, пропуск символа, добавление нового символа и ошибка в символе.

Алгоритмы имеют некоторое количество модификаций, позволяющих эффективнее решать поставленную задачу.

