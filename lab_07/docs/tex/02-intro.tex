{\centering \chapter*{ВВЕДЕНИЕ}}
\addcontentsline{toc}{chapter}{ВВЕДЕНИЕ}

Ассоциативные контейнеры обеспечивают быстрый доступ к данным по ключу. 
К ассоциативным контейнерам относятся: словари, словари с дубликатами, множества, множества с дубликатами и битовые \mbox{множества \cite{Shuikova2016}}. 

Словарь построен на основе пар значений. 
Первое значение пары --- ключ для идентификации элементов, второе --- собственно элемент. 
Например, в телефонном справочнике номеру телефона соответствует фамилия абонента. 
В словарях элементы хранятся в отсортированном по ключу виде. 
Поэтому для ключей должно быть определено отношение <<меньше>>. 
В словаре, в отличие от словаря с дубликатами, все ключи являются \mbox{уникальными \cite{Shuikova2016}}.

Лингвистической называется переменная, значениями которой являются слова или предложения естественного или искусственного языка \cite{Vasin2000}. 
Так переменная <<прибыль>> будет являться лингвистической, если ее значения будут не числовыми $(0,~1,~2,~3,~...,~100~\text{у.~е.})$, а лингвистическими, например:
\begin{enumerate}
\item[1)] планируемая --- значение лингвистической переменной <<прибыль>> находится в пределах плана;
\item[2)] низкая --- прибыль ниже планируемой;
\item[3)] высокая --- прибыль выше планируемой;
\item[4)] очень низкая --- прибыль значительно ниже планируемой;
\item[5)] очень высокая --- значение лингвистической переменной <<прибыль>> значительно выше планируемой.
\end{enumerate}