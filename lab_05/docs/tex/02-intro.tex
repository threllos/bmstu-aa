{\centering \chapter*{ВВЕДЕНИЕ}}
\addcontentsline{toc}{chapter}{ВВЕДЕНИЕ}

Конвейеризация --- архитектурный прием, используемый в современных процессорах с целью повышения быстродействия.
Идея заключается в разделении обработки компьютерной команды на последовательность независимых стадий с сохранением результатов в конце каждой стадии.
Обычно для выполнения каждой инструкции требуется осуществить некоторое количество однотипных операций, например, получение инструкции, раскодирование инструкции и т.п. 
Каждую из этих операций сопоставляют одной ступени конвейера. 
Некоторые современные процессоры имеют более 30 ступеней в конвейере, что повышает производительность процессора, но, однако, приводит к увеличению длительности простоя (например, в случае ошибки в предсказании условного перехода).
Не существует единого мнения по поводу оптимальной длины конвейера: различные программы могут иметь различные требования \cite{conv}.

Кластеризация --- объединение в группы схожих объектов является одной из фундаментальных задач в области анализа данных.
Список прикладных областей, где она применяется, широк: сегментация изображений, маркетинг, борьба с мошенничеством, прогнозирование, анализ текстов и многие другие \cite{intro-2}.

В рамках данной лабораторной работы конвеерная обработка данныъ будет исследоваться на алгорите кластеризации DBSCAN.