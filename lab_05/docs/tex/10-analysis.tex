\chapter{Аналитическая часть}

\section{Цели и задачи}

Цель работы: получить навыки организации асинхронного взаимодействия потоков на примере реализации алгоритма DBSCAN.

Задачи лабораторной работы:
\begin{enumerate}
	\item[1)] изучить идею конвейерной обработки данных.
	\item[2)] реализовать алгоритм DBSCAN с использованием конвейерной обработки данных;
	\item[3)] провести сравнительный анализ времени работы алгоритма для различного количества используемых потоков;
	\item[4)] обосновать полученные результаты в отчете.
\end{enumerate}

\section{Конвейер}

Конвейер --- способ организации вычислений, используемый в современных процессорах и контроллерах с целью повышения их производительности \cite{conv}. 

Конвейерная обработка данных основана на разделении получаемого трафика от контроллера (управляющего устройства) и параллельном выполении обработки получаемой информации.
Стоит отметить, что конвейеры работают параллельно, однако они могут быть связаны между собой.

\section{Алгоритм кластеризации DBSCAN}

Алгоритм DBSCAN, плотностный алгоритм для кластеризации пространственных данных с присутствием шума, был предложен Мартином Эстер, Гансом-Питером Кригель и коллегами в 1996 году как решение проблемы разбиения данных на кластеры произвольной формы.
Большинство алгоритмов, производящих плоское разбиение, создают кластеры по форме близкие к сферическим, так как минимизируют расстояние документов до центра кластера.
Авторы DBSCAN экспериментально показали, что их алгоритм способен распознать кластеры различной формы, например, как на \mbox{рисунке \ref{img:dbscan} \cite[197]{book}}.
\includeimage{dbscan}{f}{H}{.8\textwidth}{Примеры кластеров произвольной формы}

Идея, положенная в основу алгоритма, заключается в том, что внутри каждого кластера наблюдается типичная плотность точек, которая заметно выше, чем плотность снаружи кластера, а также плотность в областях с шумом ниже плотности любого из кластеров.
Ещё точнее, что для каждой точки кластера её соседство заданного радиуса должно содержать не менее некоторого числа точек, это число точек задаётся пороговым значением. Перед изложением алгоритма дадим необходимые определения.

\textit{Eps-соседство точки $p$}, обозначаемое как $N_{Eps}(p)$ \ref{for:t1} определяется как множество документов, находящихся от точки $p$ на расстояния не более $Eps$.
\begin{equation}
\label{for:t1}
 N_{Eps}(p) = \{q \in D | dist(p, q) \leq Eps\}
\end{equation}
Поиска точек, чьё $N_{Eps}(p)$ содержит хотя бы минимальное число точек ($MinPt$) не достаточно, так как точки бывают двух видов: ядровые и граничные.

Точка $p$ \textit{непосредственно плотно-достижима} из точки $p$ (при заданных $Eps$ и $MinPt$), если
$p \in N_{Eps}(p)$ и $|N_{Eps}(p)| \geq MinPt$.

Точка $p$ \textit{плотно-достижима} из точки $q$ (при заданных $Eps$ и $MinPt$), если существует последовательность точек \ref{for:t2} непосредственно плотно-достижимы из $p_i$.
\begin{equation}
\label{for:t2}
	q = p_1, p_2, \ldots, p_n = p: p_{i+1}
\end{equation}
Это отношение транзитивно, но не симметрично в общем случае, однако симметрично для двух ядровых точек.

Точка $p$ \textit{плотно-связана} с точкой $q$ (при заданных $Eps$ и $MinPt$), если существует точка $o$: $p$ и $q$ плотно-достижимы из $o$ (при заданных $Eps$ и $MinPt$).

\textit{Кластер} $C_j$ (при заданных $Eps$ и $MinPt$) --- это не пустое подмножество документов, удовлетворяющее следующих условиям:
\begin{enumerate}
	\item[1)] $\forall p, q:$ если $p \in C_j$ и $q$ плотно-достижима из $p$ (при заданных $Eps$ и $MinPt$), то; $q \in C_j$;
	\item[2)] $\forall p, q:$ $p$ плотно-связана с $q$ (при заданных $Eps$ и $MinPt$).
\end{enumerate}

\textit{Шум} --- это подмножество документов, которые не принадлежат ни одному кластеру \ref{for:t3}.
\begin{equation}
\label{for:t3}
	\{p \in D | \forall j: p \notin C_j, j = \overline{1, |C|}\}
\end{equation}

Алгоритм DBSCAN для заданных значений параметров $Eps$ и $MinPt$ исследует кластер следующим образом: сначала выбирает случайную точку, являющуюся ядровой, в качестве затравки, затем помещает в кластер саму затравку и все точки, плотно-достижимые из неё.

\section*{Вывод}

В данном разделе были рассмотрены понятия конвейерной обработки данных и алгоритм DBSCAN.