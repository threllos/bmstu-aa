\chapter*{Введение}
\addcontentsline{toc}{chapter}{Введение}

В данной лабораторной работе будут рассмотрены алгоритмы умножения матриц.
В программирование, как и в математике, часто приходится прибегать к использованию матриц.
В настоящее время умножение матриц активно используется в компьютерной графике, криптографии.

Над матрицами существует различные операции, например: сложение, возведение в степень, умножение.
В данной лабораторной работе пойдёт речь о умножении матриц и оптимизации этой операции.
Матрицы $A$ и $B$ могут быть перемножены, если число столбцов матрицы $A$ равно числу строк $B$.

Алгоритм Копперсмита-Винограда -- алгоритм умножения квадратных матриц, предложенный в 1987 году Д. Копперсмитом и Ш. Виноградом.

В данной работе будут предложены реализации следующих алгоритмов:
\begin{itemize}
	\item стандартный алгоритм умножения матриц;
	\item алгоритм Винограда;
	\item оптимизированный алгоритм Винограда.
\end{itemize}
