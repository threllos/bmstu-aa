{\centering \chapter*{ЗАКЛЮЧЕНИЕ}}
\addcontentsline{toc}{chapter}{ЗАКЛЮЧЕНИЕ}

В рамках данной лабораторной работы была достигнута поставленная цель: было изучено понятие конвейерной обработки данных и с помощью организации асинхронного взаимодействия потоков реализован алгоритм DBSCAN.

Решены все поставленные задачи:
\begin{enumerate}
	\item[1)] изучена идея конвейерной обработки данных.
	\item[2)] реализован алгоритм DBSCAN с использованием конвейерной обработки данных;
	\item[3)] проведен сравнительный анализ времени работы алгоритма для различного количества используемых потоков;
	\item[4)] обоснован полученный результат в отчете.
\end{enumerate}

Исходя из результатов, полученных в исследовательской части, при большом количестве данных, наболее эффективной является многопоточная реализация алгоритма DBSCAN.
При этом, если требуеться обработать небольшое количество данных, лучше использовать последовательную реализацию.
