\chapter{Конструкторская часть}

В данном разделе представлены схемы и трудоемкости реализуемых алгоритмов.

\section{Разработка алгоритмов}

На рисунках \ref{img:comb}, \ref{img:insert} и \ref{img:pancake} представлены схемы алгоритмов сортировки расчёской, вставками и блинной соответственно.
На рисунке \ref{img:pancake-alt} представлены схемы дополнительных алгоритмов, использующихся в блинной сортировке.

\img{170mm}{comb}{Схема алгоритма сортировки расчёской}
\img{170mm}{insert}{Схема алгоритма сортировки вставками}
\img{170mm}{pancake}{Схема алгоритма сортировки блинной}
\clearpage
\img{170mm}{pancake-alt}{Схемы алгоритмов поиска максимального индекса и перестановки элементов в массиве}
\clearpage

\section{Модель вычислений}

Для последующего вычисления трудоемкости необходимо ввести модель вычислений:
\begin{enumerate}
    \item операции из списка (\ref{for:opers}) имеют трудоемкость 1;
        \begin{equation}
            \label{for:opers}
            +, -, =, +=, -=, ==, !=, <, >, <=, >=, [], ++, {-}-
        \end{equation}
    \item операции из списка (\ref{for:opers2}) имеют трудоемкость 2;
        \begin{equation}
            \label{for:opers2}
            *, /, \%, *=, /=, \%=
        \end{equation}
    \item трудоемкость оператора выбора \code{if условие then A else B} рассчитывается, как (\ref{for:if});
        \begin{equation}
            \label{for:if}
            f_{if} = f_{\text{условия}} +
            \begin{cases}
            f_A, & \text{если условие выполняется,}\\
            f_B, & \text{иначе.}
            \end{cases}
        \end{equation}
    \item трудоемкость цикла рассчитывается, как (\ref{for:for});
        \begin{equation}
            \label{for:for}
            f_{for} = f_{\text{инициализации}} + f_{\text{сравнения}} + N(f_{\text{тела}} + f_{\text{инкремента}} + f_{\text{сравнения}})
        \end{equation}
	\item трудоемкость вызова функции равна 0.
\end{enumerate}


\section{Трудоёмкость алгоритмов}

Обозначим во всех последующих вычислениях размер массивов как N.

\subsection{Алгоритм сортировки расчёской}

Рассмотрим трудоемкость реализации алгоритма сортировки расчёской.

Трудоёмкость сортировки расчёской состоит из:
\begin{itemize}
	\item предварительных расчётов с трудоёмкостью $f(N) = 1 + 2$
	\item двойного цикла, трудоёмкость которого равна 
	$f(N) = \underbrace{1}_{\text{срав}} + log_t(N) 
	\cdot 
	(\underbrace{1}_{f_\text{иниц}} + \underbrace{2}_{f_\text{срав}} + 9N + \underbrace{1}_{f_\text{инк}}) 
	+ \underbrace{2}_{f_\text{деления}}$,
\end{itemize}
где t -- фактор, в нашем случае 1.247. 

Трудоёмкость при лучшем случае (отсортированный массив): 
$f(N) = log_t(N) \cdot (4 + N) + 3$.

Однако стоит учитывать, что при факторе $\approx$ 1.3, трудоёмкость 
аппроксимируется след. образом: 
$log_t(N) \cdot (4 + N) + 3 \approx C \cdot log(N) \cdot 
N \approx O(N \cdot log(N))$, где $C$ -- некая константа. 

В работе \cite{byte_ocr} сообщается о абстрактной аппроксимации и просьбе быть ``оптимистичными''. 
Наличие в рассуждении ``просьбы'' является минусом, так как читателю придётся взять на ``веру'' сказанное.
В работе \cite{comb_cocktail_counting_sort_compare} ведётся рассуждение о выводе худшего случая экспериментальным
путём, однако стоит сказать, что экспериментальный путь не является гарантией полученной асимптотики.
В работе \cite{kolmogorov_complexity} вводится математический расчёт нижней границы худшего случая. В данной 
работе приводится огромное количество аппроксимаций.
В работе \cite{simd} ведётся разговор о приближении асимптотики худшего случая.
В работе \cite{dobosiewicz_shaker_sort} приводится матемаческий расчёт асимпотики среднего случая, 
указание о том, что средний случай рассчитать сложнее, чем худший.
Доказательство асимптотики среднего и худшего случая.

Стоит обратить внимание, что в перечисленных работах происходит расчёт худшего случая, но не 
приводятся конкретные примеры.

Трудоёмкость при худшем случае: $f(N) =log_t(N) \cdot (4 + 9N) + 3 \approx O(N^2)$.

\subsection{Алгоритм сортировки вставками}

Рассмотрим трудоемкость реализации алгоритма сортировки вставками.

Алгоритм сортировки вставками состоит только из двойного цикла, соответственно
$f_{\text{``вставок''}} = f_{\text{цикла с вложенностью 2}}$.

Рассчитаем трудоёмкость для вложенного цикла:
$f_{\text{цикла с вложенностью 2}} = 1 + 1 + N \cdot 
(1 + 6 + \frac{N + 1}{2} \cdot 8 + 1) + 1$.

Трудоёмкость при лучшем случае (отсортированный массив). Алгоритм ни разу не войдет 
во внутренний цикл:
$3 + N \cdot 
(\frac{N + 1}{2} \cdot 0 + 8) = 3 + 8N \approx O(N)$

Трудоёмкость при худшем случае (отсортированный в обратном порядке массив).
Алгоритм будет полностью проходить внутренний цикл каждый раз:
$3 + N \cdot 
(8 + \frac{N + 1}{2} \cdot 8) = 3 + 8N + 8N \cdot \frac{N + 1}{2} \cdot 8 \approx O(N^2)$.


\subsection{Алгоритм блинной сортировки}

Рассмотрим трудоемкость реализации алгоритма блинной сортировки.

Этот алгоритм сортировки состоит только из двойного цикла, соответственно
$f_{\text{``блинная''}} = f_{\text{цикла с вложенностью 2}}$.

Рассчитаем трудоёмкость для вложенного цикла:
$f_{\text{цикла с вложенностью 2}} = 1 + 1 + N \cdot 
(3 + 5 \сdot N + 1 + 1 + 4 \cdot 2 \cdot N) + 1$.

Трудоёмкость при лучшем случае (алгоритм ни разу не сделает переворот):
$3 + N \cdot 
(\frac{N + 1}{2} \cdot 0 + 8) = 3 + 4N + 5N^2 \approx O(N^2)$

Трудоёмкость при худшем случае (отсортированный в обратном порядке массив).
Алгоритм будет полностью проходить внутренний цикл каждый раз:
$3 + N \cdot 
(8 + \frac{N + 1}{2} \cdot 8) = 3 + 9N + 13N^2 \approx O(N^2)$.

Стоит отметить, что асимптотики худшего и лучшего случаев совпадает.

\section*{Вывод}

На основе формул и теоретических данных, полученных в аналитическом разделе, были спроектированы схемы алгоритмов.
Для каждого из них были рассчитаны и оценены лучшие и худшие случаи.
