\chapter{Технологическая часть}

В данном разделе будут приведены требования к программному обеспечению, средства реализации и листинга кода.

\section{Требования к ПО}

Программа должна отвечать следующим требованиям:
\begin{itemize}
	\item программа на вход принимает две строки;
	\item программа выдает редакционное расстояние для всех четырех методов. 
\end{itemize}

\section{Средства реализации}

В качестве языка программирования для реализации данной лабораторной работы был выбран современный компилируемый ЯП Golang \cite{golang}.
Данный выбор обусловлен моим желанием расширить свои знания в области применения данного языка, а также тем, что данный язык предоставляет широкие возможности для написания тестов \cite{gotest}.

\section{Листинг кода}

В листингах \ref{lst:l-iter} -- \ref{lst:dl-recurs-cash} приведены листинги описанных алгоритмов Левентштейна и Дамерау-Левентштейна.
В листинге \ref{lst:some} приведены вспомогательные функции.
В листингах \ref{lst:tests} и \ref{lst:benches} приведены примеры реализации тестов и бенчмарков.

\clearpage
\begin{lstinputlisting}[
	caption={Итеративный алгоритм Левентштейна},
	label={lst:l-iter},
	style={golang}
]{./assets/listing/l-iter.lst}
\end{lstinputlisting}

\clearpage
\begin{lstinputlisting}[
	caption={Итеративный алгоритм Дамерау-Левентштейна},
	label={lst:dl-iter},
	style={golang}
]{./assets/listing/dl-iter.lst}
\end{lstinputlisting}

\clearpage
\begin{lstinputlisting}[
	caption={Рекурсивный алгоритм Дамерау-Левентштейна},
	label={lst:dl-recurs},
	style={golang}
]{./assets/listing/dl-recurs.lst}
\end{lstinputlisting}

\clearpage
\begin{lstinputlisting}[
	caption={Рекурсивный алгоритм с кэшем Дамерау-Левентштейна},
	label={lst:dl-recurs-cash},
	style={golang}
]{./assets/listing/dl-recurs-cash.lst}
\end{lstinputlisting}

\clearpage
\begin{lstinputlisting}[
	caption={Вспомогательные функции для расчёта расстояний},
	label={lst:some},
	style={golang}
]{./assets/listing/some.lst}
\end{lstinputlisting}

\clearpage
\begin{lstinputlisting}[
	caption={Пример реализации тестов},
	label={lst:tests},
	style={golang}
]{./assets/listing/tests-part1.lst}
\end{lstinputlisting}

\clearpage
\begin{lstinputlisting}[
	style={golang}
]{./assets/listing/tests-part2.lst}
\end{lstinputlisting}

\clearpage
\begin{lstinputlisting}[
	caption={Пример реализации бенчмарка},
	label={lst:benches},
	style={golang}
]{./assets/listing/benches.lst}
\end{lstinputlisting}

\clearpage
\section{Тестирование функций}

В таблице \ref{tbl:test} приведены тесты для функций, реализующих алгоритмы вычисления расстояний Левентштейна и Дамерау-Левентштейна.
Все тесты пройдены успешно.

\begin{table}[h!]
	\begin{center}
		\caption{\label{tbl:test}Тестовые данные}
		\begin{tabular}{|c|c|c|c|}
			\hline
			\bfseries Строка 1 & \bfseries Строка 2 & \bfseries Левентштейн & \bfseries Дамерау-Левентштейн \\ 
			\hline
			cook & cooker & 2 & 2 \\ \hline
			aboba & aboba & 0 & 0 \\ \hline
			абвгдеё & абвг & 3 & 3 \\ \hline
			"" & qwer & 4 & 4 \\ \hline % как пустую строку написать
			qwerty & ytrewq & 6 & 5 \\ \hline
			qwerty & wqreyt & 4 & 3 \\ \hline
		\end{tabular}
	\end{center}
\end{table}

\section*{Вывод}

На основе схем из конструкторского раздела были разработаны и протестированы спроектированные алгоритмы.
