\chapter{Аналитическая часть}

\section{Цели и задачи}

Цель работы: получить навыки формализации разделенных диапозоном значений величин на категории.

Задачи лабораторной работы:
\begin{enumerate}
	\item[1)] формализовать объект и его признак;
	\item[2)] составить анкету для ее заполнения респондетами;
	\item[3)] провести анкетирование;
	\item[4)] описать 3 формата запроса от пользователя;
	\item[5)] реализовать функцию принадлежности термам числовых значений признака;
	\item[6)] привести примеры запросов пользователя.
\end{enumerate}

\section{Словарь}

Словарь --- абстрактный тип данных, позволяющий хранить пары вида <<ключ-значение>> и поддерживающий операции добавления, поиска и удаления пары по ключу. 
В паре $(key,~value)$ значение $value$ называется значением, ассоциированным с ключом $key$. 
Поиск --- основная задача при использовании словаря, которая может решаться различными способами.

\section{Алгоритм бинарного поиска в словаре}

Важным условием алгоритма бинарного поисква в словаре является упорядоченность  ключа в последовательности, в данном случае ключом является цена видеокарты.
Идея бинарного поиска заключается в делении на части массива по значению ключу в середине.
Если ключ у экземпляра массива, расположенного в середине, больше чем, входной ключ, то далее происходит идентичная обработка с середины и до конца массива, иначе --- с начала до середины.
В результате работы алгоритма пользователь получает экземпляр массива, ключ которого равен входному \cite{bins}.

\section{Формализация объекта и его признака}

Объектами в текущей лабораторной работе являются видеокарты. 
Признаком является цена видеокарты, которая в рамках данной задачи измеряется в тысячах рублей. 
Словарь используется для описания обьекта <<видеокарта>> со следующими параметрами: ключ --- терм (словесное описание признака), значение --- числовые значения признака (цена видеокарты в тысячах рублей). 
Доступные термы:

\begin{itemize}[label=---]
	\item[1)] очень дешевая;
	\item[2)] дешевая;
	\item[3)] средняя;
	\item[4)] дорогая;
	\item[5)] очень дорогая.
\end{itemize}

Доступные числовые значения признака: от 20 тысяч рублей до 150 тысяч рублей.

\section{Вопросы}

Программное обеспечение должно будет отвечать на следюущие вопросы.
\begin{itemize}[label=---]
	\item Какие видеокарты являются дорогими?
	\item Можешь перечислить все дешевые видеокарты?
	\item Какая цена у очень дорогих видеокарт?
	\item Можешь вывести все видеокарты?
\end{itemize}

\section*{Вывод}

Был формализован объект с признаком, а также рассмотрен алгоритм бинарного поиска в словаре.
