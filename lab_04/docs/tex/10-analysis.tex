\chapter{Аналитическая часть}

\section{Цели и задачи}

Цель работы: получить навыки организации многопоточных вычислений на примере реализации алгоритма DBSCAN.

Задачи лабораторной работы:
\begin{enumerate}
	\item[1)] изучить понятия параллельных вычислений;
	\item[2)] реализовать алгоритм DBSCAN с использованием параллельных вычислений;
	\item[3)] провести сравнительный анализ времени работы алгоритма для различного количества используемых потоков;
	\item[4)] обосновать полученные результаты в отчете.
\end{enumerate}

\section{Поток}

Поток --- это основная единица, которой операционная  система выделяет время процессора.
Каждый поток имеет приоритет  планирования и набор структур, в которых система сохраняет контекст  потока, когда выполнение потока приостановлено. Контекст потока  содержит все сведения, позволяющие потоку безболезненно возобновить выполнение, в том числе набор регистров процессора и стек потока.

Выделяют 2 типа потоков: нативные и зеленые.
Зелёные потоки --- потоки, управление которыми производит виртуальная машина, нативные потоки --- управление которыми производит операционная система.

Так называемые зеленые потоки лишь эмулируют многопоточную среду, не полагаясь на возможности операционной системы, что позволяет им работать без поддержки встроенных потоков.
Однако зеленые потоки, как и любой инструмент имеет свои недостатки.
Один из которых --- блокировка всего процесса.
Если на одном из потоков происходит блокировка, то блокируется весь процесс, ввиду того операционная система не знает о существовании зеленых потоков \cite{threads}.

\section{Алгоритм кластеризации DBSCAN}

Алгоритм DBSCAN, плотностный алгоритм для кластеризации пространственных данных с присутствием шума, был предложен Мартином Эстер, Гансом-Питером Кригель и коллегами в 1996 году как решение проблемы разбиения данных на кластеры произвольной формы.
Большинство алгоритмов, производящих плоское разбиение, создают кластеры по форме близкие к сферическим, так как минимизируют расстояние документов до центра кластера.
Авторы DBSCAN экспериментально показали, что их алгоритм способен распознать кластеры различной формы, например, как на \mbox{рисунке \ref{img:dbscan} \cite[197]{book}}.
\includeimage{dbscan}{f}{H}{.8\textwidth}{Примеры кластеров произвольной формы}

Идея, положенная в основу алгоритма, заключается в том, что внутри каждого кластера наблюдается типичная плотность точек, которая заметно выше, чем плотность снаружи кластера, а также плотность в областях с шумом ниже плотности любого из кластеров.
Ещё точнее, что для каждой точки кластера её соседство заданного радиуса должно содержать не менее некоторого числа точек, это число точек задаётся пороговым значением. Перед изложением алгоритма дадим необходимые определения.

\textit{Eps-соседство точки $p$}, обозначаемое как $N_{Eps}(p)$ определяется как множество документов, находящихся от точки $p$ на расстояния не более $Eps$:
$N_{Eps}(p) = \{q \in D | dist(p, q) \leq Eps\}$.
Поиска точек, чьё $N_{Eps}(p)$ содержит хотя бы минимальное число точек ($MinPt$) не достаточно, так как точки бывают двух видов: ядровые и граничные.

Точка $p$ \textit{непосредственно плотно-достижима} из точки $p$ (при заданных $Eps$ и $MinPt$), если
$p \in N_{Eps}(p)$ и $|N_{Eps}(p)| \geq MinPt$.

Точка $p$ \textit{плотно-достижима} из точки $q$ (при заданных $Eps$ и $MinPt$), если существует последовательность точек
$q = p_1, p_2, \ldots, p_n = p: p_{i+1}$
непосредственно плотно-достижимы из $p_i$.
Это отношение транзитивно, но не симметрично в общем случае, однако симметрично для двух ядровых точек.

Точка $p$ \textit{плотно-связана} с точкой $q$ (при заданных $Eps$ и $MinPt$), если существует точка $o$: $p$ и $q$ плотно-достижимы из $o$ (при заданных $Eps$ и $MinPt$).

\textit{Кластер} $C_j$ (при заданных $Eps$ и $MinPt$) --- это не пустое подмножество документов, удовлетворяющее следующих условиям:
\begin{enumerate}
	\item[1)] $\forall p, q:$ если $p \in C_j$ и $q$ плотно-достижима из $p$ (при заданных $Eps$ и $MinPt$), то; $q \in C_j$;
	\item[2)] $\forall p, q:$ $p$ плотно-связана с $q$ (при заданных $Eps$ и $MinPt$).
\end{enumerate}

\textit{Шум} --- это подмножество документов, которые не принадлежат ни одному кластеру: $\{p \in D | \forall j: p \notin C_j, j = \overline{1, |C|}\}$.

Алгоритм DBSCAN для заданных значений параметров $Eps$ и $MinPt$ исследует кластер следующим образом: сначала выбирает случайную точку, являющуюся ядровой, в качестве затравки, затем помещает в кластер саму затравку и все точки, плотно-достижимые из неё.

\section*{Вывод}

В данном разделе были рассмотрены понятия параллельных вычислений и алгоритм DBSCAN.