{\centering \chapter*{ВВЕДЕНИЕ}}
\addcontentsline{toc}{chapter}{ВВЕДЕНИЕ}

Применение параллельных вычислительных систем (ПВС) является стратегическим направлением развития вычислительной техники.
Это обстоятельство вызвано не только принципиальным ограничением максимально возможного быстродействия обычных последовательных ЭВМ, но и практически постоянным наличием вычислительных задач, для решения которых возможностей существующих средств вычислительной техники всегда оказывается недостаточно \cite{intro-1}.

Кластеризация --- объединение в группы схожих объектов является одной из фундаментальных задач в области анализа данных.
Список прикладных областей, где она применяется, широк: сегментация изображений, маркетинг, борьба с мошенничеством, прогнозирование, анализ текстов и многие другие \cite{intro-2}.

В рамках данной лабораторной работы параллельные вычисления будут исследоваться на алгорите кластеризации DBSCAN.