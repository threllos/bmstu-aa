\chapter{Аналитическая часть}

\section{Цели и задачи}

Цель работы: получить навыки разработки алгоритмов сортировки на трех примерах.

Задачи лабораторной работы:
\begin{enumerate}
	\item[1)] изучить 3 алгоритма сортировки: расческой, вставками, блинная;
	\item[2)] реализовать изученные алгоритмы сортировки;
	\item[3)] провести сравнительный анализ трудоемкости алгоритмов на основе теоретических расчетов и выбранной модели вычислений;
	\item[4)] провести сравнительный анализ времени работы алгоритмов для различных размеров входного массива;
	\item[5)] обосновать полученные результаты.
\end{enumerate}

\section{Алгоритмы сортировки}

\subsection{Сортировка расческой}
Сортировка расческой улучшает сортировку пузырьком, и конкурирует с алгоритмами, подобными быстрой сортировке.
Основная идея — устранить черепах, или маленькие значения в конце списка, которые крайне замедляют сортировку пузырьком (кролики, большие значения в начале списка, не представляют проблемы для сортировки пузырьком).
В сортировке пузырьком, когда сравниваются два элемента, промежуток (расстояние друг от друга) равен 1.
Основная идея сортировки расческой в том, что этот промежуток может быть гораздо больше, чем единица.

\subsection{Сортировка вставками}
Сортировка вставками — алгоритм сортировки, котором элементы входной последовательности просматриваются по одному, и каждый новый поступивший элемент размещается в подходящее место среди ранее упорядоченных элементов.

В начальный момент отсортированная последовательность пуста.
На каждом шаге алгоритма выбирается один из элементов входных данных и помещается на нужную позицию в уже отсортированной последовательности до тех пор, пока набор входных данных не будет исчерпан.
В любой момент времени в отсортированной последовательности элементы удовлетворяют требованиям к выходным данным алгоритма.

\subsection{Сортировка блинная}
Единственная операция, допустимая в алгоритме сортировки — переворот элементов последовательности до какого-либо индекса.
В отличие от традиционных алгоритмов, в которых минимизируют количество сравнений, в блинной сортировке требуется сделать как можно меньше переворотов.
Процесс можно визуально представить как стопку блинов, которую тасуют путем взятия нескольких блинов сверху и их переворачивания.

\section*{Вывод}
Были рассмотрены три различных алгоритма: сортировка расческой, сортировка вставками и блинная сортировка.
