\chapter{Исследовательская часть}


Для формирования системы запросов о цене видеокарты, стало необходимым провести опрос среди респондентов и построить функцию принадлежности термам числовых значений признака, описываемого лингвистической переменной.

В данном разделе приведена анкета, отправленная респондентам.
Также представлены результаты анкетирования и обработки мнений респондентов.

\section{Анкета для респондентов}


Таблица \ref{tbl:form} представляет из себя анкету, отправленную респондентам.

\begin{table}[H]
	\centering
	\caption{Анкета, отправленная респондентам}
	\label{tbl:form}
	\begin{tabular}{|c|c|c|c|c|c|c|c|c|}
		\hline
	 	\multirow{2}{*}{Респондент} & \multirow{2}{*}{Терм} & \multicolumn{7}{c|}{Цена видеокарты в тыс. рублей} \\
	 	\cline{3-9} && 20~ & 40~ & 60~ & 85~ & 110 & 130 & 150 \\
	  	\hline
	  	\multirow{5}{*}{1} & очень дешевая &&&&&&&  \\
	  	\cline{2-9} & дешевая &&&&&&&  \\
	  	\cline{2-9} & средняя &&&&&&&  \\
	  	\cline{2-9} & дорогая &&&&&&&  \\
	  	\cline{2-9} & очень дорогая &&&&&&&  \\
	  	\hline
	\end{tabular}
\end{table}

\section{Результаты анкетирования}

В таблице \ref{tbl:resps} указаны данные о респондетах и их номера.
В таблице \ref{tbl:results} указаны результаты анкетирования респондентов.

\begin{table}[H]
	\centering
	\caption{Информация о респондентах}
	\label{tbl:resps}
	\begin{tabular}{|c|c|}
	  \hline
	  Фамилия и имя & Номер \\
	  \hline
	  Волгина Ольга & 1 \\
	  \hline
	  Худяков Владимир & 2 \\
	  \hline
	  Морозов Дмитрий & 3 \\
	  \hline
	  Загайнов Никита & 4 \\
	  \hline
	  Нарандаев Дамир & 5 \\
	  \hline
	\end{tabular}
\end{table}

\begin{table}[H]
	\centering
	\caption{Результаты анкетирования}
	\label{tbl:results}
	\begin{tabular}{|c|c|c|c|c|c|c|c|c|}
		\hline
	 	\multirow{2}{*}{Респондент} & \multirow{2}{*}{Терм} & \multicolumn{7}{c|}{Цена видеокарты в тыс. рублей} \\
	 	\cline{3-9} && 20~ & 40~ & 60~ & 85~ & 110 & 130 & 150 \\ \hline
  
	  	\multirow{5}{*}{1} & очень дешевая &1&0&0&0&0&0&0  \\
	  	\cline{2-9}
	  	& дешевая &0&1&1&1&0&0&0  \\
	  	\cline{2-9}
	  	& средняя &0&0&0&0&1&0&0  \\
	  	\cline{2-9}
	  	& дорогая &0&0&0&0&1&0&0  \\
	  	\cline{2-9}
	  	& очень дорогая &0&0&0&0&0&1&1  \\
	  	\hline
  
		\multirow{5}{*}{2} & очень дешевая &1&0&0&0&0&0&0  \\
		\cline{2-9}
		& дешевая &0&1&0&0&0&0&0  \\
		\cline{2-9}
		& средняя &0&0&1&1&0&0&0  \\
		\cline{2-9}
		& дорогая &0&0&0&1&1&1&0  \\
		\cline{2-9}
		& очень дорогая &0&0&0&0&0&0&1  \\
		\hline
		
		\multirow{5}{*}{3} & очень дешевая &1&0&0&0&0&0&0  \\
		\cline{2-9}
		& дешевая &0&1&0&0&0&0&0  \\
		\cline{2-9}
		& средняя &0&0&1&1&0&0&0  \\
		\cline{2-9}
		& дорогая &0&0&0&0&1&1&0  \\
		\cline{2-9}
		& очень дорогая &0&0&0&0&0&1&1  \\
		\hline
  
		\multirow{5}{*}{4} & очень дешевая &1&1&0&0&0&0&0  \\
		\cline{2-9}
		& дешевая &0&1&1&0&0&0&0  \\
		\cline{2-9}
		& средняя &0&0&0&1&0&0&0  \\
		\cline{2-9}
		& дорогая &0&0&0&0&1&0&0  \\
		\cline{2-9}
		& очень дорогая &0&0&0&0&0&1&1  \\
		\hline
	
		\multirow{5}{*}{5} & очень дешевая &1&0&0&0&0&0&0  \\
		\cline{2-9}
		& дешевая &0&1&1&0&0&0&0  \\
		\cline{2-9}
		& средняя &0&0&0&1&0&0&0  \\
		\cline{2-9}
		& дорогая &0&0&0&0&1&1&0  \\
		\cline{2-9}
		& очень дорогая &0&0&0&0&0&0&1  \\
		\hline
  
	\end{tabular}
\end{table}

\section{Функция принадлежности}

На рисунке \ref{plt:res} представлена зависимость принадлежности от цены видеокарты для каждого терма.

\begin{figure}[H]
	\begin{tikzpicture}
		\begin{axis}[
			legend pos = north west,
			grid = major,
			xlabel = {$x$ -- цена видеокарты, тыс. рублей},
			ylabel = $y$ -- принадлежность,
			height = 0.4\paperheight,
			width = 0.75\paperwidth,
		]
		\legend{
			лёгкий,
			умеренный,
			средний,
			тяжёлый,
			тяжеленный,
		}
		\addplot table[col sep=space]{assets/csv/1.csv};
		\addplot table[col sep=space]{assets/csv/2.csv};
		\addplot table[col sep=space]{assets/csv/3.csv};
		\addplot table[col sep=space]{assets/csv/4.csv};
		\addplot table[col sep=space]{assets/csv/5.csv};
		\end{axis}
	\end{tikzpicture}
	\caption{Графики зависимости принадлежности от цены видеокарты}
	\label{plt:res}
\end{figure}

\section{Соответствие признаков и диапазонов значений}

В таблице \ref{tbl:prizn} приведено соответствие признаков и диапазонов цен видеокарт.

\begin{table}[H]
	\centering
	\caption{Соответствие признаков и диапазонов цен видеокарт}
	\label{tbl:prizn}
	\begin{tabular}{ |c|c| }
		\hline
	  	Признак 		& Диапазон\\
	  	\hline
	  	очень дешевая	& [20; 31]\\
	  	\hline
	  	дешевая 		& [31; 66]\\
	  	\hline
	  	средняя 		& [66; 98]\\
	  	\hline
	  	дорогая 		& [98; 130]\\
	  	\hline
	  	очень дорогая 	& [130; 150]\\
	  	\hline
	\end{tabular}
\end{table}

\section*{Вывод}

В текущем разделе было проведено анкетирование респондентов. 
По результатам опроса была построена функция принадлежности термам числовых значений признака, описываемого лингвистической переменной, на основе статистической обработки мнений респондентов, выступающих в роли экспертов.