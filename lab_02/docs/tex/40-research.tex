\chapter{Исследовательская часть}

В данном разделе будут приведены примеры работы программы, постановка эксперимента и сравнительный анализ алгоритмов на основе полученных данных.

\section{Технические характеристики}

Тестирование выполнялось на устройстве со следующими техническими характеристиками:
\begin{itemize}
	\item Операционная система: Windows 11 x64 \cite{windows}.
	\item Память: 8 GiB.
	\item Процессор: AMD Ryzen 5 3550H \cite{amd}.
\end{itemize}

Замеры проводились на ноутбуке, включенном в сеть электропитания.
Во время тестирования ноутбук был нагружен только встроенными приложениями окружения, окружением, а также непосредственно системой тестирования.

\section{Пример работы программы}

% TODO: сделать скрин, а до этого переделать прогу
На рисунке \ref{img:preview} представлен результат работы программы.

\img{50mm}{preview}{Пример работы программы}

\section{Время выполнения алгоритмов}

Результаты тестирования приведены в таблице \ref{tbl:times}.
На рисунках \ref{plt:iter-rec} и \ref{plt:l-dl} приведены зависимости времени работы алгоритма от длины строк. 

% TODO: посмотреть замеры и со своими сверить
\begin{table}[h!]
	\begin{center}
		\caption{Время работы алгоритмов}
		\label{tbl:times}
		\begin{tabular}{|c|c|c|c|}
			\hline
			& \multicolumn{3}{c|}{\bfseries Время работы, нс} \\ \cline{2-4}
			\bfseries Размерность $n \times n$ & \bfseries Классический & \bfseries Виноград & \bfseries Виноград опт. 
			\csvreader{assets/csv/times.csv}{}
			{\\\hline \csvcoli&\csvcolii&\csvcoliii&\csvcoliv}
			\\\hline
		\end{tabular}
	\end{center}
\end{table}

\begin{figure}[h!]
	\centering
	\begin{tikzpicture}
		\begin{axis}[
			axis lines=left,
			xlabel=Размер массива,
			ylabel={Время, нс},
			legend pos=north west,
			ymajorgrids=true
		]
			\addplot table[x=len,y=classic,col sep=comma]{assets/csv/times.csv};
			\addplot table[x=len,y=win,col sep=comma]{assets/csv/times.csv};
			\addplot table[x=len,y=win-opt,col sep=comma]{assets/csv/times.csv};
			\legend{Классический, Виноград, Виноград опт.}
		\end{axis}
	\end{tikzpicture}
	\captionsetup{justification=centering}
	\caption{Временные характеристики}
	\label{plt:times}
\end{figure}
\clearpage

% TODO: переписать
Результаты тестирования приведены в таблице \ref{tbl:times}.
На рисунках \ref{plt:iter-rec} и \ref{plt:l-dl} приведены зависимости времени работы алгоритма от длины строк. 

\begin{table}[h!]
	\begin{center}
		\caption{Время работы алгоритмов при нечетной размерности}
		\label{tbl:times-odd}
		\begin{tabular}{|c|c|c|c|}
			\hline
			& \multicolumn{3}{c|}{\bfseries Время работы, нс} \\ \cline{2-4}
			\bfseries Размерность $n \times n$ & \bfseries Классический & \bfseries Виноград & \bfseries Виноград опт. 
			\csvreader{assets/csv/times-odd.csv}{}
			{\\\hline \csvcoli&\csvcolii&\csvcoliii&\csvcoliv}
			\\\hline
		\end{tabular}
	\end{center}
\end{table}

\begin{figure}[h!]
	\centering
	\begin{tikzpicture}
		\begin{axis}[
			axis lines=left,
			xlabel=Размер массива,
			ylabel={Время, нс},
			legend pos=north west,
			ymajorgrids=true
		]
			\addplot table[x=len,y=classic,col sep=comma]{assets/csv/times-odd.csv};
			\addplot table[x=len,y=win,col sep=comma]{assets/csv/times-odd.csv};
			\addplot table[x=len,y=win-opt,col sep=comma]{assets/csv/times-odd.csv};
			\legend{Классический, Виноград, Виноград опт.}
		\end{axis}
	\end{tikzpicture}
	\captionsetup{justification=centering}
	\caption{Временные характеристики на нечетных размерах матриц}
	\label{plt:times-odd}
\end{figure}

\section*{Вывод}
% TODO: попиздеть вывод у кого-то
В данном разделе было произведено сравнение количества затраченного вре­мени вышеизложенных алгоритмов.
Наименее затратным по времени оказался оптимизированный алгоритм Винограда.
Но при этом ему дополнительно требуется $n+m$ памяти под результат (т.е. $n+m$ прибавляется к тому количеству памяти, которое выделяется под результат в реализации стандарного алгоритма умножения матриц - $n \cdot m$).

Время работы реализации алгоритма Винограда незначительно меньше времени работы реализации простого алгоритма умножения, однако при размере $800 \times 800$ время вычислений реализации алгоритма Винограда на 0.3 секунды меньше, нежели у реализации простого алгоритма (что составляет в данном случае порядка $10\%$).

Такой результат совпадает с теоретически полученными оценками трудоемкости алгоритмов 