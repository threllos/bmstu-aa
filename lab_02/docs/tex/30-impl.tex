\chapter{Технологическая часть}

В данном разделе будут приведены требования к программному обеспечению, средства реализации и листинга кода.

\section{Требования к ПО}

Программа должна отвечать следующим требованиям:
\begin{itemize}
	\item программа на вход принимает две матрицы $A$ и $B$;
	\item количество столбцов матрицы $A$ должно быть равно количеству строк матрицы $B$;
	\item программа выдает результат умножения введенных пользователем матриц. 
\end{itemize}

\section{Средства реализации}

В качестве языка программирования для реализации данной лабораторной работы был выбран современный компилируемый ЯП Golang \cite{golang}.
Данный выбор обусловлен моим желанием расширить свои знания в области применения данного языка, а также тем, что данный язык предоставляет широкие возможности для написания тестов \cite{gotest}.

\section{Листинг кода}

В листингах \ref{lst:mul-class}, \ref{lst:mul-win} и \ref{lst:mul-win-opt} приведены листинги описанных алгоритмов умножения матриц.
В листингах \ref{lst:win-alt} и \ref{lst:win-opt-alt} приведены вспомогательные функции, использующиеся в алгоритмах Винограда.
В листингах \ref{lst:tests} и \ref{lst:benches} приведены примеры реализации тестов и бенчмарков.

\clearpage
\begin{lstinputlisting}[
	caption={Классический алгоритм умножения матриц},
	label={lst:mul-class},
	style={golang}
]{./assets/listing/mul-class.lst}
\end{lstinputlisting}

\clearpage
\begin{lstinputlisting}[
	caption={Умножение матриц алгоритмом Винограда},
	label={lst:mul-win},
	style={golang}
]{./assets/listing/mul-win.lst}
\end{lstinputlisting}

\begin{lstinputlisting}[
	caption={Вспомогательные функии для алгоритма Винограда},
	label={lst:win-alt},
	style={golang}
]{./assets/listing/win-alt.lst}
\end{lstinputlisting}

\clearpage
\begin{lstinputlisting}[
	caption={Умножение матриц оптимизированным алгоритмом Винограда},
	label={lst:mul-win-opt},
	style={golang}
]{./assets/listing/mul-win-opt.lst}
\end{lstinputlisting}

\begin{lstinputlisting}[
	caption={Вспомогательные функии для оптимизированного алгоритма},
	label={lst:win-opt-alt},
	style={golang}
]{./assets/listing/win-opt-alt.lst}
\end{lstinputlisting}

\clearpage
\begin{lstinputlisting}[
	caption={Пример реализации тестов},
	label={lst:tests},
	style={golang}
]{./assets/listing/tests.lst}
\end{lstinputlisting}

\clearpage
\begin{lstinputlisting}[
	caption={Пример реализации бенчмарка},
	label={lst:benches},
	style={golang}
]{./assets/listing/benches.lst}
\end{lstinputlisting}

\clearpage
\section{Тестирование функций}

В таблице \ref{tbl:test} приведены тесты для функций, реализующих умножение матриц.
Все тесты пройдены успешно.

\begin{table}[h!]
	\begin{center}
		\caption{\label{tbl:test}Тестовые данные}
		\begin{tabular}{|c|c|c|}
			\hline
			\bfseries Первая матрица& \bfseries Вторая матрица & \bfseries Ожидаемый результат  \\ 
			\hline

			$\begin{pmatrix}
			5
			\end{pmatrix}$ &
			$\begin{pmatrix}
			5
			\end{pmatrix}$ &
			$\begin{pmatrix}
			10
			\end{pmatrix}$ \\ \hline

			$\begin{pmatrix}
			5 \\
			5 
			\end{pmatrix}$ &
			$\begin{pmatrix}
			2 & 3
			\end{pmatrix}$ &
			$\begin{pmatrix}
			10 & 15\\
			10 & 15
			\end{pmatrix}$ \\ \hline

			$\begin{pmatrix}
			1 & 2 & 3\\
			4 & 5 & 6
			\end{pmatrix}$ &
			$\begin{pmatrix}
			2 & 4\\
			6 & 8\\
			10 & 12
			\end{pmatrix}$ &
			$\begin{pmatrix}
			44 & 56\\
			98 & 128
			\end{pmatrix}$ \\ \hline

			$\begin{pmatrix}
			-1 & -2 & -3\\
			-4 & -5 & -6\\
			-7 & -8 & -9
			\end{pmatrix}$ &
			$\begin{pmatrix}
			1 & 2 & 3\\
			4 & 5 & 6\\
			7 & 8 & 9
			\end{pmatrix}$ &
			$\begin{pmatrix}
			-30 & -36 & -42\\
			-66 & -81 & -96\\
			-102 & -126 & -150
			\end{pmatrix}$ \\ \hline

		\end{tabular}
	\end{center}
\end{table}

\section*{Вывод}

На основе схем из конструкторского раздела были разработаны и протестированы спроектированные алгоритмы.
