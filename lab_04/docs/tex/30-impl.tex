\chapter{Технологическая часть}

\section{Требования к программному обеспечению}

Программа должна отвечать следующим требованиям:
\begin{itemize}[label=---]
	\item программа получает на вход файл с точками, минимальное количество точек в кластере, минимальное расстояние и количество потоков;
	\item программа выдает в результате файл с кластеризованными точками;
	\item алгоритм реализован с помощью параллельных вычислений;
	\item программа должна измерять реальное время.
\end{itemize}

\section{Средства реализации}

В качестве языка программирования для реализации данной лабораторной работы был выбран \textit{C\texttt{++}} ввиду следующих причин:
\begin{itemize}[label=---]
	\item мое желанием расширить свои знания в области применения данного языка;
	\item возможность измерять реальное время выполнения алгоритма;
	\item данный язык предоставляет широкие возможности для работы с потоками.
\end{itemize}

Таким образом, с помощью языка программирования \textit{C\texttt{++}} можно реализовать программное обеспечение, которое соответствует перечисленным выше требованиям.

\section{Реализация алгоритма}

В листинге \ref{lst:dbscan.lst} приведена реализация алгоритма DBSCAN.
В листинге \ref{lst:claster.lst} приведена функция кластеризации, используемая в внешнем циклке DBSCAN.
\clearpage
\includelistingpretty{dbscan.lst}{c}{Реализация алгоритма DBSCAN}
\clearpage	
\includelistingpretty{claster.lst}{c}{Функция кластеризации}
\clearpage

\section{Тестовые данные}

В таблице \ref{tbl:test} приведены тесты для функции, реализующей алгоритм DBSCAN.
Тесты выполнялись по методологии черного ящика.
Все тесты пройдены успешно.

\begin{table}[H]
	\begin{center}
		\caption{\label{tbl:test}Тестовые данные}
		\begin{tabular}{|c|c|c|c|}
			\hline
			\bfseries Точки & \bfseries MinPt & \bfseries Eps & \bfseries Количество кластеров  \\ 
			\hline

			1 1 & \multirow{3}{*}{3} & \multirow{3}{*}{0.5} & \multirow{3}{*}{1} \\
			1 2 & & & \\
			2 1 & & & \\ \hline

			1 1 & \multirow{3}{*}{3} & \multirow{3}{*}{1.5} & \multirow{3}{*}{0} \\
			1 2 & & & \\
			2 1 & & & \\ \hline

			1 1 & \multirow{7}{*}{3} & \multirow{7}{*}{1} & \multirow{7}{*}{2} \\
			1 2 & & & \\
			2 1 & & & \\
			5 5 & & & \\
			5 6 & & & \\
			6 5 & & & \\
			10 10 & & & \\ \hline

		\end{tabular}
	\end{center}
\end{table}

\section*{Вывод}

На основе схемы, полученной в конструкторском разделе, был реализован и протестирован алгоритм DBSCAN.
