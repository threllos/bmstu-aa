\chapter{Конструкторская часть}

\section{Разработка алгоритмов}

На рисунках \ref{img:comb}, \ref{img:insert} и \ref{img:pancake} представлены схемы алгоритмов сортировки расчёской, вставками и блинной соответственно.
На рисунке \ref{img:pancake-alt} представлены схемы дополнительных алгоритмов, использующихся в блинной сортировке.

\img{170mm}{comb}{Схема алгоритма сортировки расчёской}
\img{170mm}{insert}{Схема алгоритма сортировки вставками}
\img{170mm}{pancake}{Схема алгоритма сортировки блинной}
\clearpage
\img{170mm}{pancake-alt}{Схемы алгоритмов поиска максимального индекса и перестановки элементов в массиве}
\clearpage

\section{Модель вычислений}

Для последующего вычисления трудоемкости необходимо ввести модель вычислений:
\begin{enumerate}
    \item операции из списка (\ref{for:opers}) имеют трудоемкость 1;
        \begin{equation}
            \label{for:opers}
            +, -, /, *, =, +=, -=, *=, /=, ==, !=, <, >, <=, >=, [], ++, {-}-
        \end{equation}
    \item трудоемкость оператора выбора \code{if условие then A else B} рассчитывается, как (\ref{for:if});
	\begin{equation}
        \label{for:if}
        f_{if} = f_{\text{условия}} +
        \begin{cases}
        f_A, & \text{если условие выполняется,}\\
        f_B, & \text{иначе.}
        \end{cases}
	\end{equation}
\item трудоемкость цикла рассчитывается, как (\ref{for:for});
    \begin{equation}
        \label{for:for}
        f_{for} = f_{\text{инициализации}} + f_{\text{сравнения}} + N(f_{\text{тела}} + f_{\text{инкремента}} + f_{\text{сравнения}})
    \end{equation}
	\item трудоемкость вызова функции равна 0.
\end{enumerate}


\section{Трудоёмкость алгоритмов}

Обозначим во всех последующих вычислениях размер массивов как N.

\subsection{Алгоритм сортировки расчёской}

Трудоёмкость алгоритма сортировки расчёской состоит из:
\begin{itemize}
    \item Трудоёмкость сравнения и инкремента внешнего цикла, которая равна (\ref{for:comb_outer}):
        \begin{equation}
            \label{for:comb_outer}
            f_{outer} = 2 + 2(N - 1)
        \end{equation}
    \item Трудоёмкость внутреннего цикла, количество итераций которых меняется в промежутке $[0..N-gap]$, которая равна (\ref{for:comb_inner}):
        \begin{equation}
            \label{for:comb_inner}
            f_{inner} = 3(N - 1) + \frac{N \cdot (N - 1)}{2} \cdot (3 + f_{if})
        \end{equation}
    \item Трудоёмкость условия во внутреннем цикле, которая равна (\ref{for:comb_if}):
        \begin{equation}
            \label{for:comb_if}
            f_{if} = 4 + \begin{cases}
                0, & \text{л.с.}\\
                10, & \text{х.с.}\\
            \end{cases}
        \end{equation}
\end{itemize}

Трудоёмкость в лучшем случае (\ref{for:comb_best}):
\begin{equation}
    \label{for:comb_best}
    f_{best} = -3 + \frac{3}{2} N + \frac{7}{2} N^2 \approx \frac{7}{2} N^2 = O(N^2)
\end{equation}

Трудоёмкость в худшем случае (\ref{for:comb_worst}):
\begin{equation}
    \label{for:comb_worst}
    f_{worst} = -3 - 8N + 8N^2 \approx 8N^2 = O(N^2)
\end{equation}

\subsection{Алгоритм сортировки выбором}

Трудоёмкость алгоритма сортировки выбором состоит из:
\begin{itemize}
    \item Трудоёмкость сравнения, инкремента внешнего цикла, а также зависимых только от него операций, по $i \in [1..N)$, которая равна (\ref{for:selection_outer}):
        \begin{equation}
            \label{for:selection_outer}
            f_{outer} = 2 + 12(N - 1)
        \end{equation}
    \item Суммарная трудоёмкость внутренних циклов, количество итераций которых меняется в промежутке $[1..N-1]$, которая равна (\ref{for:selection_inner}):
        \begin{equation}
            \label{for:selection_inner}
            f_{inner} = 2(N - 1) + \frac{N \cdot (N - 1)}{2} \cdot f_{if}
        \end{equation}
    \item Трудоёмкость условия во внутреннем цикле, которая равна (\ref{for:selection_if}):
        \begin{equation}
            \label{for:selection_if}
            f_{if} = 3 + \begin{cases}
                0, & \text{л.с.}\\
                3, & \text{х.с.}\\
            \end{cases}
        \end{equation}
\end{itemize}

Трудоёмкость в лучшем случае (\ref{for:selection_best}):
\begin{equation}
    \label{for:selection_best}
    f_{best} = -12 + 12.5N + \frac{3}{2}N^2 \approx \frac{3}{2}N = O(N^2)
\end{equation}

Трудоёмкость в худшем случае (\ref{for:selection_worst}):
\begin{equation}
    \label{for:selection_worst}
    f_{worst} = -12 + 11N + 3N^2 \approx 3N^2 = O(N^2)
\end{equation}


\subsection{Алгоритм сортировки вставками}

Трудоёмкость сортировки вставками может быть рассчитана таким же образом, что и трудоёмкости алгоритмов выше. Асимптотическая трудоёмкость алгоритма сортировки вставками $O(N^2)$ в худшем случае и $O(N)$ --- в лучшем.

\section*{Вывод}

Были разработаны схемы всех трех алгоритмов сортировки.
Для каждого из них были рассчитаны и оценены лучшие и худшие случаи.
