\chapter*{Введение}
\addcontentsline{toc}{chapter}{Введение}

% TODO: опять много теории в введении
Нахождение редакционного расстояния -- одна из задач компьютерной лингвистики, которая находит применение в огромном количестве областей, например:
\begin{itemize}
	\item исправление ошибок в тексте в поисковых запросах;
    \item сравнение текстовых файлов утилитой diff \cite{diff};
    \item сравнение генов, хромосом и белков в биоинформатике.
\end{itemize}

Впервые задачу поставил советский ученый В. И. Левенштейн при изучении последовательностей 0-1 \cite{levenshtein}.
Впоследствии более общую задачу для произвольного алфавита связали с его именем.
Позже Фредерик Дамерау заявил, что при исследовании орфографических ошибок в информационно-поисковых системах более 80\% человеческих ошибок при наборе текстов составляют перестановки соседних символов, пропуск символа, добавление нового символа и ошибка в символе.

\textbf{Расстояние Левенштейна} -- метрика, измеряющая разность двух строк символов, определяемая в количестве редакторских операций (а именно удаления, вставки и замены), требуемых для преобразования одной последовательности в другую.

\textbf{Расстояние Дамерау-Левенштейна} -- модификация, добавляющая к редакторским операциям транспозицию (обмен двух соседних символов местами).

Алгоритмы имеют некоторое количество модификаций, позволяющих эффективнее решать поставленную задачу.
В данной работе будут предложены реализации следующих алгоритмов:
\begin{itemize}
	\item нерекурсивный метод поиска расстояния Левентштейна;
    \item нерекурсивный метод поиска Дамерау-Левентштейна;
    \item рекурсивный метод поиска Дамерау-Левентштейна;
    \item рекурсивный с кешированием метод поиска Дамерау-Левентштейна.
\end{itemize}

\clearpage

Задачи лабораторной работы:
\begin{itemize}
	\item изучение алгоритмов редакционных расстояний Левенштейна и Дамерау-Левенштейна;
    \item получение практических навыков реализаций алгоритмов редакционных расстояний Левенштейна и Дамерау-Левенштейна;
    \item проведение сравнительного анализа алгоритмов определения расстояния между строками по затратам времени и памяти;
	\item применение метода динамического программирования для реализации алгоритмов;
	\item описание и обоснование полученных результатов в отчете о выполненной лабораторной работе. 
\end{itemize}
