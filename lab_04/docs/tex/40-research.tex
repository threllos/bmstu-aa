\chapter{Исследовательская часть}

\section{Технические характеристики}

Тестирование выполнялось на устройстве со следующими техническими характеристиками.
\begin{enumerate}
	\item Операционная система: Ubuntu Linux 64-bit.
	\item Память: 8 ГБ.
	\item Процессор: AMD Ryzen 5 3550H.
\end{enumerate}

\section{Время выполнения алгоритма}

Время работы функции замерено с помощью ассемблерной инструкции \textbf{rdtsc}, которая читает счетчик Time Stamp Counter и возвращает 64-битное количество тиков с момента последнего сброса процессора.

Результаты тестирования приведены в таблице \ref{tbl:times}.
На рисунке \ref{plt:times} приведена зависимость времени работы алгоритма от количества точек. 

\begin{table}[H]
	\begin{center}
		\caption{Время работы алгоритма}
		\label{tbl:times}
		\begin{tabular}{|c|c|c|c|}
			\hline
			& \multicolumn{3}{c|}{\bfseries Время работы, тики} \\ \cline{2-4}
			\bfseries Количество точек & \bfseries 1 поток & \bfseries 2 потока & \bfseries 4 потока 
			\csvreader{assets/csv/times.csv}{}
			{\\\hline \csvcoli&\csvcolii&\csvcoliii&\csvcoliv}
			\\\hline
		\end{tabular}
	\end{center}
\end{table}

\begin{figure}[H]
	\centering
	\begin{tikzpicture}
		\begin{axis}[
			axis lines=left,
			xlabel=Количество точек,
			ylabel={Время, нс},
			legend pos=north west,
			ymajorgrids=true
		]
			\addplot table[x=len,y=thr1,col sep=comma]{assets/csv/times.csv};
			\addplot table[x=len,y=thr2,col sep=comma]{assets/csv/times.csv};
			\addplot table[x=len,y=thr4,col sep=comma]{assets/csv/times.csv};
			\legend{1 поток, 2 потока, 4 потока}
		\end{axis}
	\end{tikzpicture}
	\captionsetup{justification=centering}
	\caption{Зависимость времени выполнения от количества потоков}
	\label{plt:times}
\end{figure}

\section*{Вывод}

Согласно полученным при проведении эксперимента данным, при малом количестве входных данных, последовательно работающий алгоритм работает быстрее, чем многопоточные. 
Это объясняется тем, что на выделение поток тратиться значительное количество ресурсов.
С увеличением входных данных, многопоточные начинают обгонять последовательный алгоритм, поскольку время, затраченное на выделение потоков, становиться незначительным, по сравнению со временем, затраченным на вычисления.