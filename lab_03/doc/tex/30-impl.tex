\chapter{Технологическая часть}

В данном разделе приведены средства реализации и листинги кода.

\section{Требования к ПО}

К программе предъявляется ряд требований:
\begin{itemize}
	\item на вход подаётся массив сравнимых элементов;
	\item на выходе — тот же массив, но в отсортированном порядке.
\end{itemize}

\section{Средства реализации}

В качестве языка программирования для реализации данной лабораторной работы был выбран современный компилируемый ЯП Golang \cite{golang}.
Данный выбор обусловлен моим желанием расширить свои знания в области применения данного языка, а также тем, что данный язык предоставляет широкие возможности для написания тестов \cite{gotest}.

\section{Листинг кода}

В листингах \ref{lst:comb} -- \ref{lst:pancake} приведены листинги алгоритма сортировки расчёской, вставками и блинной соответственно.
В листингах \ref{lst:tests} -- \ref{lst:benches} приведены примеры реализации тестов и бенчмарков.

\clearpage
\begin{lstinputlisting}[
	caption={Алгоритм сортировки расчёской},
	label={lst:comb},
	style={golang}
]{./assets/listing/comb.lst}
\end{lstinputlisting}

\begin{lstinputlisting}[
	caption={Алгоритм сортировки вставками},
	label={lst:insert},
	style={golang}
]{./assets/listing/insert.lst}
\end{lstinputlisting}

\clearpage
\begin{lstinputlisting}[
	caption={Алгоритм блинной сортировки},
	label={lst:pancake},
	style={golang}
]{./assets/listing/pancake.lst}
\end{lstinputlisting}

\clearpage
\begin{lstinputlisting}[
	caption={Пример реализации теста},
	label={lst:tests},
	style={golang}
]{./assets/listing/test.lst}
\end{lstinputlisting}

\clearpage
\begin{lstinputlisting}[
	caption={Пример реализации бенчмарка},
	label={lst:benches},
	style={golang}
]{./assets/listing/bench.lst}
\end{lstinputlisting}

\section{Тестирование функций}

В таблице~\ref{tbl:test} приведены тесты для функций, реализующих алгоритмы сортировки. Тесты пройдены успешно.

\begin{table}[h!]
	\begin{center}
		\caption{\label{tbl:test}Тестовые данные}
		\begin{tabular}{|c|c|c|}
			\hline
			Входной массив & Ожидаемый результат & Результат \\ 
			\hline
			$[1,3,5,2,4]$ & $[1,2,3,4,5]$  & $[1,2,3,4,5]$\\
			$[1,2,3,4,5]$  & $[1,2,3,4,5]$ & $[1,2,3,4,5]$\\
			$[5,4,3,2,1]$  & $[1,2,3,4,5]$  & $[1,2,3,4,5]$\\
			$[5,-1,4,3,2]$  & $[-1,2,3,4,5]$  & $[-1,2,3,4,5]$\\
			$[777]$  & $[777]$  & $[777]$\\
			$[]$  & $[]$  & $[]$\\
			\hline
		\end{tabular}
	\end{center}
\end{table}

\section*{Вывод}

Правильный выбор инструментов разработки позволил эффективно реализовать алгоритмы, настроить модульное тестирование и выполнить исследовательский раздел лабораторной работы.
