\chapter{Технологическая часть}

\section{Требования к программному обеспечению}

Программа должна отвечать следующим требованиям:
\begin{itemize}[label=---]
	\item программа получает на вход файл с матрицей смежности;
	\item программа выдает в результате кратчайший путь и массив вершин, по которым он строится;
	\item для вычисления кратчайшего пути используются алгоритма полного перебора и муравьиного алгоритма;
	\item программа должна измерять реальное время.
\end{itemize}

Программа должна обрабатывать ошибки (например, отсутствие файла с матрицей расстояний) и корректно завершать работу с выводом информации об ошибке на экран.
 
\section{Средства реализации}

В качестве языка программирования для реализации данной лабораторной работы был выбран \textit{Python} ввиду следующих причин:
\begin{itemize}[label=---]
	\item мое желанием расширить свои знания в области применения данного языка;
	\item возможность измерять реальное время выполнения алгоритма;
	\item в языке реализован модуль \textit{numpy}, который позволяет работать с матрицами.
\end{itemize}

Таким образом, с помощью языка \textit{Python} можно реализовать программное обеспечение, которое соответствует перечисленным выше требованиям.

\section{Реализация алгоритма}

В листинге~\ref{lst:tsp.lst} показана реализация алгоритма полного перебора для решения задачи коммивояжера.
В листинге~\ref{lst:aco.lst} показана реализация муравьиного алгоритма для решения задачи коммивояжера.
\includelistingpretty{aco.lst}{python}{Реализация муравьиного алгоритма}
\pagebreak
\includelistingpretty{tsp.lst}{python}{Реализация алгоритма полного перебора}

\section{Тестовые данные}

В таблице~\ref{tbl:tests} приведены тестовые данные для двух функций, реализующих алгоритмы для решения задачи коммивояжера (поиска гамильтонова цикла). 
Результата записаны в следующем формате: значение кратчайшего пути; кратчайший путь.

Тесты выполнялись по методологии черного ящика (модульное тестирование). 
Все тесты пройдены успешно.

\begin{table}[H]
	\begin{center}
		\caption{Результаты тестирования}
		\label{tbl:tests}
		\begin{tabular}{|c|c|c|}
			\hline
			Матрица расстояний & Полный перебор & Муравьиный алгоритм \\
			\hline
				$\begin{pmatrix}
				0
				\end{pmatrix}$ & 0: [1, 1] & 0: [1, 1] \\ \hline
				$\begin{pmatrix}
				0 & 10 \\
				10 & 0 \\
				\end{pmatrix}$ & 10: [1, 2, 1] & 10: [1, 2, 1] \\ \hline
				$\begin{pmatrix}
				0 & 10 & 15 & 20 \\
				10 & 0 & 35 & 25 \\
				15 & 35 & 0 & 30 \\
				20 & 25 & 30 & 0 \\
				\end{pmatrix}$ & 80: [1, 2, 4, 3, 1] & 80: [1, 2, 4, 3, 1] \\
			\hline
		\end{tabular}
	\end{center}
\end{table}

\section*{Вывод}

В текущем разделе был написан исходный код алгоритма полного перебора и муравьиного алгоритма для решения задачи коммивояжера. 
Описаны тесты и приведены результаты тестирования.