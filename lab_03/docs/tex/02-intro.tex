\chapter*{Введение}
\addcontentsline{toc}{chapter}{Введение}
\textbf{Сортировка} -- это процесс разделения объектов по виду или сорту, программисты традиционно используют это слово в гораздо более узком смысле, обозначая им такую перестановку предметов, при которой они располагаются в порядке возрастания или убывания\cite{knut}.
Такой процесс следовало бы называть не сортировкой, а \textit{упорядочением}, но использование этого слова привело бы к путанице из-за перегруженности значениями слова \textit{порядок}.

Алгоритмы сортировки используются практически в любой программной системе.
Целью алгоритмов сортировки является упорядочение последовательности элементов данных.
Поиск элемента в последовательности отсортированных данных занимает время, пропорциональное логарифму количеству элементов в последовательности, а поиск элемента в последовательности не отсортированных данных занимает время, пропорциональное количеству элементов в последовательности, то есть намного больше.
Существует множество различных методов сортировки данных.
Однако любой алгоритм сортировки можно разбить на три основные части:
\begin{itemize}
    \item сравнение, определяющее упорядоченность пары элементов;
    \item перестановка, меняющая местами пару элементов;
    \item собственно сортирующий алгоритм, который осуществляет сравнение и перестановку элементов данных до тех пор, пока все эти элементы не будут упорядочены.
\end{itemize}

Важнейшей характеристикой любого алгоритма сортировки является скорость его работы, которая определяется функциональной зависимостью среднего времени сортировки последовательностей элементов данных, заданной длины, от этой длины. Время сортировки будет пропорционально количеству сравнений и перестановки элементов данных в процессе их сортировки.

\clearpage
Задачи лабораторной работы:
\begin{itemize}
	\item изучить и реализовать 3 алгоритма сортировки: расчёской, вставками, блинная;
	\item провести сравнительный анализ трудоёмкости алгоритмов на основе теоретических расчетов и выбранной модели вычислений;
	\item провести сравнительный анализ алгоритмов на основе экспериментальных данных;
    \item подготовить отчета по лабораторной работе.
\end{itemize}
