\chapter{Исследовательская часть}

\section{Технические характеристики}

Тестирование выполнялось на устройстве со следующими техническими характеристиками.
\begin{enumerate}
	\item Операционная система: Windows 11 x64 \cite{windows}.
	\item Память: 8 GiB.
	\item Процессор: AMD Ryzen 5 3550H \cite{amd}.
\end{enumerate}

Замеры проводились на ноутбуке, включенном в сеть электропитания.
Во время тестирования ноутбук был нагружен только встроенными приложениями окружения, окружением, а также непосредственно системой тестирования.

\section{Время выполнения алгоритмов}

Результаты замеров приведены в таблицах \ref{tbl:sorted}, \ref{tbl:reversed} и \ref{tbl:random}.
На рисунках \ref{plt:sorted}, \ref{plt:reversed} и \ref{plt:random} приведены графики зависимостей времени работы алгоритмов сортировки от размеров массивов на отсортированных, обратно отсортированных и случайных данных.
\clearpage

\begin{table}[H]
	\begin{center}
		\begin{tabular}{|c|c|c|c|}
			\hline
			                 & \multicolumn{3}{c|}{\bfseries Время сортировки, нс}           \\ \cline{2-4}
			\bfseries Размер & \bfseries Расческой & \bfseries Вставками & \bfseries Блинная
			\csvreader{assets/csv/sorted.csv}{}
			{\\\hline \csvcoli&\csvcolii&\csvcoliii&\csvcoliv}
			\\\hline
		\end{tabular}
	\end{center}
	\caption{Время работы алгоритмов сортировки на отсортированных данных}
	\label{tbl:sorted}
\end{table}

\begin{figure}[H]
	\centering
	\begin{tikzpicture}
		\begin{axis}[
			axis lines=left,
			xlabel=Размер,
			ylabel={Время, нс},
			legend pos=north west,
			ymajorgrids=true
		]
			\addplot table[x=size,y=comb,col sep=comma]{assets/csv/sorted.csv};
			\addplot table[x=size,y=insert,col sep=comma]{assets/csv/sorted.csv};
			\addplot table[x=size,y=pancake,col sep=comma]{assets/csv/sorted.csv};
			\legend{Расческой, Вставками, Блинная}
		\end{axis}
	\end{tikzpicture}
	\captionsetup{justification=centering}
	\caption{Зависимость времени работы алгоритма сортировки от размера отсортированного массива}
	\label{plt:sorted}
\end{figure}

\begin{table}[H]
	\begin{center}
		\begin{tabular}{|c|c|c|c|}
			\hline
			                 & \multicolumn{3}{c|}{\bfseries Время сортировки, нс}           \\ \cline{2-4}
			\bfseries Размер & \bfseries Расческой & \bfseries Вставками & \bfseries Блинная
			\csvreader{assets/csv/reversed.csv}{}
			{\\\hline \csvcoli&\csvcolii&\csvcoliii&\csvcoliv}
			\\\hline
		\end{tabular}
	\end{center}
	\caption{Время работы алгоритмов сортировки на обратно отсортированных данных}
	\label{tbl:reversed}
\end{table}

\begin{figure}[H]
	\centering
	\begin{tikzpicture}
		\begin{axis}[
			axis lines=left,
			xlabel=Размер,
			ylabel={Время, нс},
			legend pos=north west,
			ymajorgrids=true
		]
			\addplot table[x=size,y=comb,col sep=comma]{assets/csv/reversed.csv};
			\addplot table[x=size,y=insert,col sep=comma]{assets/csv/reversed.csv};
			\addplot table[x=size,y=pancake,col sep=comma]{assets/csv/reversed.csv};
			\legend{Расческой, Вставками, Блинная}
		\end{axis}
	\end{tikzpicture}
	\captionsetup{justification=centering}
	\caption{Зависимость времени работы алгоритма сортировки от размера массива, отсортированного в обратном порядке}
	\label{plt:reversed}
\end{figure}

\begin{table}[H]
	\begin{center}
		\begin{tabular}{|c|c|c|c|}
			\hline
			                 & \multicolumn{3}{c|}{\bfseries Время сортировки, нс}           \\ \cline{2-4}
			\bfseries Размер & \bfseries Расческой & \bfseries Вставками & \bfseries Блинная
			\csvreader{assets/csv/random.csv}{}
			{\\\hline \csvcoli&\csvcolii&\csvcoliii&\csvcoliv}
			\\\hline
		\end{tabular}
	\end{center}
	\caption{Время работы алгоритмов сортировки на случайных данных}
	\label{tbl:random}
\end{table}

\begin{figure}[H]
	\centering
	\begin{tikzpicture}
		\begin{axis}[
			axis lines=left,
			xlabel=Размер,
			ylabel={Время, нс},
			legend pos=north west,
			ymajorgrids=true
		]
			\addplot table[x=size,y=comb,col sep=comma]{assets/csv/random.csv};
			\addplot table[x=size,y=insert,col sep=comma]{assets/csv/random.csv};
			\addplot table[x=size,y=pancake,col sep=comma]{assets/csv/random.csv};
			\legend{Расческой, Вставками, Блинная}
		\end{axis}
	\end{tikzpicture}
	\captionsetup{justification=centering}
	\caption{Зависимость времени работы алгоритма сортировки от размера случайного массива}
	\label{plt:random}
\end{figure}

\section*{Вывод}

В данном разделе были сравнены реализованные алгоритмы по времени.
Алгоритм сортировки расческой работает лучше остальных при случайном расположении элементов в массиве.
