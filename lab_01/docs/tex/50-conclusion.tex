\chapter*{Заключение}
\addcontentsline{toc}{chapter}{Заключение}

В рамках лабораторной работы были:

\begin{itemize}
	\item изучены алгоритмы редакционных расстояний Левенштейна и Дамерау-Левенштейна;
	\item получены практические навыки реализации данных алгоритмов;
	\item получены навыки динамического программирования;
	\item проведены анализы затрат работы программы по времени и по памяти. 
\end{itemize}

По итогу реализации алгоритмов поиска редакционного расстояния итеративный способ оказался быстрее рекурсивного, но он расходует больше памяти.
Алгоритм Левенштейна работает быстрее модифицированной версии Дамерау-Левенштейна, т.к. в теле цикла выполняется меньше операций.
Улучшение рекурсивного алгоритма, добавлением кэша, значительно увеличивает скорость работы рекурсивной реализации за счёт того, что не производятся повторные вычисления.

