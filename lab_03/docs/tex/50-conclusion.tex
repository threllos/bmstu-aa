{\centering \chapter*{ЗАКЛЮЧЕНИЕ}}
\addcontentsline{toc}{chapter}{ЗАКЛЮЧЕНИЕ}

В рамках данной лабораторной работы была достигнута поставленная цель: были изучены принципы разработки алгоритмов сортировки на трех примерах.

Решены все поставленные задачи:
\begin{enumerate}
	\item[1)] изучены 3 алгоритма сортировки: расческой, вставками, блинная;
	\item[2)] реализованы изученные алгоритмы сортировки;
	\item[3)] проведен сравнительный анализ трудоемкости алгоритмов на основе теоретических расчетов и выбранной модели вычислений;
	\item[4)] проведен сравнительный анализ времени работы алгоритмов для различных размеров входного массива;
	\item[5)] обоснованы полученные результаты.
\end{enumerate}

Исходя из результатов, полученных в исследовательской части, получилось соотнести время выполнения с трудоемкостью для всех алгоритмов.
Для массивов со случайным расположением элементов наиболее эффективной реализацией из рассмотренных являеться алгоритм сортировки расческой.
