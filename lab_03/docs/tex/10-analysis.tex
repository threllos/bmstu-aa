\chapter{Аналитическая часть}

В этом разделе будут представлены описания трех алгоритмов сортировки.

\section{Сортировка расчёской}
Сортировка расчёской улучшает сортировку пузырьком, и конкурирует с алгоритмами, подобными быстрой сортировке.
Основная идея — устранить черепах, или маленькие значения в конце списка, которые крайне замедляют сортировку пузырьком (кролики, большие значения в начале списка, не представляют проблемы для сортировки пузырьком).
В сортировке пузырьком, когда сравниваются два элемента, промежуток (расстояние друг от друга) равен 1.
Основная идея сортировки расчёской в том, что этот промежуток может быть гораздо больше, чем единица.

\section{Сортировка вставками}
Сортировка вставками — алгоритм сортировки, котором элементы входной последовательности просматриваются по одному, и каждый новый поступивший элемент размещается в подходящее место среди ранее упорядоченных элементов.

В начальный момент отсортированная последовательность пуста.
На каждом шаге алгоритма выбирается один из элементов входных данных и помещается на нужную позицию в уже отсортированной последовательности до тех пор, пока набор входных данных не будет исчерпан.
В любой момент времени в отсортированной последовательности элементы удовлетворяют требованиям к выходным данным алгоритма.

\section{Сортировка блинная}
Единственная операция, допустимая в алгоритме сортировки — переворот элементов последовательности до какого-либо индекса.
В отличие от традиционных алгоритмов, в которых минимизируют количество сравнений, в блинной сортировке требуется сделать как можно меньше переворотов.
Процесс можно визуально представить как стопку блинов, которую тасуют путём взятия нескольких блинов сверху и их переворачивания.

\section*{Вывод}
Были рассмотрены три различных алгоритма: сортировка расчёской, сортировка вставками и блинная сортировка.
Необходимо провести теоретическую оценку алгоритмов и проверить ее экспериментально.
