\chapter{Конструкторская часть}

В данном разделе представлены схемы и трудоемкости реализуемых алгоритмов.

\section{Разработка алгоритмов}

На рисунке \ref{img:mul_class} и \ref{img:mul_win} представлены схемы стандартного алгоритма и алгоритма Винограда умножения матриц соответственно.
На рисунке \ref{img:mul_win-alt} представлены схемы алгоритмов предварительной обработки данных, использующихся в алгоритме Винограда.

\img{150mm}{mul_class}{Схема стандартного алгоритма умножения матриц}
\clearpage
\img{190mm}{mul_win}{Схема алгоритма Винограда умножения матриц}
\clearpage
\img{190mm}{mul_win-alt}{Схемы алгоритмов предварительной обработки данных}
\clearpage

\section{Модель вычислений}

Для последующего вычисления трудоемкости необходимо ввести модель вычислений:
\begin{enumerate}
    \item операции из списка (\ref{for:opers}) имеют трудоемкость 1;
        \begin{equation}
            \label{for:opers}
            +, -, =, +=, -=, ==, !=, <, >, <=, >=, [], ++, {-}-
        \end{equation}
    \item операции из списка (\ref{for:opers2}) имеют трудоемкость 2;
        \begin{equation}
            \label{for:opers2}
            *, /, \%, *=, /=, \%=
        \end{equation}
    \item трудоемкость оператора выбора \code{if условие then A else B} рассчитывается, как (\ref{for:if});
        \begin{equation}
            \label{for:if}
            f_{if} = f_{\text{условия}} +
            \begin{cases}
            f_A, & \text{если условие выполняется,}\\
            f_B, & \text{иначе.}
            \end{cases}
        \end{equation}
    \item трудоемкость цикла рассчитывается, как (\ref{for:for});
        \begin{equation}
            \label{for:for}
            f_{for} = f_{\text{инициализации}} + f_{\text{сравнения}} + N(f_{\text{тела}} + f_{\text{инкремента}} + f_{\text{сравнения}})
        \end{equation}
	\item трудоемкость вызова функции равна 0.
\end{enumerate}

\section{Трудоёмкость алгоритмов}

Обозначим во всех последующих вычислениях размерность матрицы $A$, как $n \times k$, а матрицы $B$, как $k \times m$.

\subsection{Стандартный алгоритм умножения матриц}

Трудоёмкость стандартного алгоритма умножения матриц включает в себя трудоемкости:
\begin{itemize}
	\item внешнего цикла по $i \in [1 \dots n]$, рассчитывающегося как (\ref{el:3top});
        \begin{equation}
            \label{el:3top}
            f_{i} = 2 + n(2 + f_{j})
        \end{equation}
	\item цикла по $j \in [1 \dots m]$, рассчитывающегося как (\ref{el:2top});
        \begin{equation}
            \label{el:2top}
            f_{j} = 2 + m(2 + f_{k})
        \end{equation}
	\item скалярного умножения двух векторов -- цикл по $k \in [1 \dots k]$, трудоёмкость которого равняется (\ref{el:1top}).
        \begin{equation}
            \label{el:1top}
            f_{j} = 2 + 14k
        \end{equation}
\end{itemize}

Трудоёмкость стандартного алгоритма равна трудоёмкости внешнего цикла, поэтому её можно вычислить как (\ref{eq:class}):
\begin{equation}
	\label{eq:class}
	f_{classic} = 2 + n(4 + m(4 + 14k)) = 2 + 4n + 4nm + 14nmk \approx 14nmk
\end{equation}

\subsection{Алгоритм Винограда}

Трудоёмкость алгоритма Винограда состоит из:
\begin{itemize}
    \item инициализация массивов $RF$ и $CF$, имеющая трудоёмкость (\ref{eq:win1});
        \begin{equation}
            \label{eq:win1}
            f_{init} = n + m
        \end{equation}
	\item заполнение массива $RF$, который представляет из себя фактор строк и имеет трудоёмкость (\ref{eq:win2});
        \begin{equation}
            \label{eq:win2}
            f_{RF} = 2 + n(4 + \frac{k}{2}(4 + 6 + 1 + 2 + 3 \cdot 2)) = 2 + 4n + 9.5nk
        \end{equation}
	\item заполнение массива $CF$, который представляет из себя фактор столбцов и имеет трудоёмкость (\ref{eq:win3});
        \begin{equation}
            \label{eq:win3}
            f_{CF} = 2 + m(4 + \frac{k}{2}(4 + 6 + 1 + 2 + 3 \cdot 2)) = 2 + 4m + 9.5mk
        \end{equation}
	\item цикла заполнения ячеек матрицы C для чётных размеров, имеющий трудоёмкость (\ref{eq:win4});
        \begin{equation}
            \label{eq:win4}
            f_{even} = 2 + n(4 + m(2 + 11 + \frac{k}{2}(4 + 28))) = 2 + 4n + 13nm + 16nmk
        \end{equation}
    \item дополнительный цикл, если размер матрицы нечётный, имеющий трудоёмкость (\ref{eq:win5}).
        \begin{equation}
            \label{eq:win5}
            f_{odd} = 3 +
            \begin{cases}
            0, & \text{размер матрицы чётный,}\\
            2 + 4n + 16nm, & \text{иначе.}
            \end{cases}
        \end{equation}
\end{itemize}

Трудоёмкость алгоритма Винограда равна сумме вышеперечисленных трудоёмкостей (\ref{eq:win}):
\begin{equation}
    \label{eq:win}
    f_{win} = f_{init} + f_{RF} + f_{CF} + f_{even} + f_{onn}
\end{equation}

Итого, трудоёмкость в лучшем случае (\ref{eq:best}):
\begin{equation}
    \label{eq:best}
    f_{best} \approx 16nmk
\end{equation}

Трудоёмкость в худшем случае (\ref{eq:worst}):
\begin{equation}
    \label{eq:worst}
    f_{worst} \approx 16nmk
\end{equation}

\subsection{Оптимизированный алгоритм Винограда}

Данный алгоритм можно оптимизировать:
\begin{enumerate}
	\item заменив выражения вида $a = a + k$ на $a += k$;
	\item заменив умножение на 2 на побитовый сдвиг;
	\item предвычислив некоторые слагаемые для алгоритма.
\end{enumerate}

Трудоёмкость инициализации массивов $RF$ и $CF$ никак не меняется.
Заполнение массивов факторов и дополнительный цикл, для нечётных размеров матрицы, меняются незначительно для итоговой оценки трудоёмкости.
Трудоёмкость цикла заполнения ячеек матрицы для чётных размеров выглядит как (\ref{eq:win-opt}):
\begin{equation}
    \label{eq:win-opt}
    f_{even} = 2 + n(4 + m(4 + 7 + \frac{k}{2}(4 + 19))) = 2 + 4n + 11nm + 11.5nmk
\end{equation}

Итого, трудоёмкость в лучшем случае (\ref{eq:best-opt}):
\begin{equation}
    \label{eq:best-opt}
    f_{best} \approx 11.5nmk
\end{equation}

Трудоёмкость в худшем случае (\ref{eq:worst-opt}):
\begin{equation}
    \label{eq:worst-opt}
    f_{worst} \approx 11.5nmk
\end{equation}

\section*{Вывод}

На основе формул и теоретических данных, полученных в аналитическом разделе, были спроектированы схемы алгоритмов.
Проведена теоретическая оценка трудоёмкости и для каждого из алгоритмов были рассчитаны и оценены лучшие и худшие случаи.
